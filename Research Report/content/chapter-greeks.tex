\chapter{The Greeks for Barrier Options}
\label{sec:greeks}

The Greeks are key sensitivities in option pricing that measure how the price of an option changes with respect to various factors such as the underlying asset price, time to maturity, volatility, and interest rates. For barrier options, the calculation of the Greeks is more complex due to the added condition of a barrier, which affects the option's price path and behavior.

For the plots in this section, the following parameters will be used:

\[
\begin{aligned}
S_0 &= 80.0 \quad \text{(spot stock price)} \\
K &= 50 \quad \text{(strike price)} \\
T &= 1.0 \quad \text{(time to maturity in years)} \\
r &= 0.1 \quad \text{(risk-free rate)} \\
\sigma &= 0.2 \quad \text{(volatility)} \\
B &= 100 \quad \text{(barrier level)}
\end{aligned}
\]




\section{Delta (\(\Delta\))}

Delta for a barrier option measures the sensitivity of the option price to changes in the underlying asset price (\(S\)). It represents the rate of change of the option's price with respect to small changes in the underlying price.
\[
\Delta = \frac{\partial \text{Option Price}}{\partial S}
\]

For \textbf{knock-in} barrier options, delta behaves similarly to that of a standard European option but is influenced by the presence of the barrier. It reflects how changes in the underlying price affect the probability of the barrier being breached and the option becoming active.

Figure~\ref{fig:delta_upout} shows the delta behavior of an up-and-out put option as a function of the underlying stock price (\(S\)). When the stock price is well below the barrier level (\(B\)), delta behaves similarly to a vanilla European put option. As the price approaches the barrier, delta tends toward zero because the likelihood of the option being knocked out increases, reducing its sensitivity.

To further analyze the delta across different barrier options, refer to the table below:


\begin{center}
	\begin{table}[H]
		\begin{tabular}{ | m{4em} | m{1.4cm}| m{1.4cm} | m{1.4cm}|} 
			\hline
			\textbf{Barrier Type}         & \textbf{Price Far from Barrier} & \textbf{Price Near the Barrier} & \textbf{Intuition}  \\
			\hline
			\textbf{Up-and-Out (Knock-Out)} & Similar to vanilla options     & $\Delta \to 0$ as $S \to B$  & Probability of knock-out increases as $S$ nears $B$. \\ 
			\textbf{Down-and-Out (Knock-Out)} & Similar to vanilla options     & $\Delta \to 0$ as $S \to B$  & Probability of knock-out increases as $S$ nears $B$. \\ 
			\hline
			\textbf{Up-and-In (Knock-In)}    & $\Delta \approx 0$ far from barrier   & $\Delta \uparrow$ as $S \to B$  & Option becomes active as the price hits the barrier. \\ 
			\hline
			\textbf{Down-and-In (Knock-In)}  & $\Delta \approx 0$ far from barrier   & $\Delta \uparrow$ as $S \to B$  & Option becomes active as the price hits the barrier. \\ 
			\hline
		\end{tabular}
		\caption{Delta behavior for different types of barrier options.}
		\label{tab:delta_barrier_options}
	\end{table}
\end{center}

\textbf{Key Intuition:}
\begin{itemize}
    \item \textbf{Knock-Out Options:} Lose their sensitivity as $\Delta$ tends toward $0$ as the likelihood of being knocked out near the barrier increases.
    \item \textbf{Knock-In Options:} Show increasing sensitivity as the underlying price nears the barrier because the probability of activation rises.
\end{itemize}

The visual analysis in Figure~\ref{fig:delta_upout} and the insights from Table~\ref{tab:delta_barrier_options} provide a comprehensive understanding of how delta behaves across different types of barrier options.
\begin{figure}[h]
    \centering
    \includegraphics[width=.65\linewidth]{content/images/delta_upout.png}
    \caption{Delta of up-and-out Put Option vs. the Stock Price.}
    \label{fig:delta_upout}
\end{figure}

\subsection{Delta Hedging}

\section{Gamma (\(\Gamma\))}

Gamma (\(\Gamma\)) for a barrier option measures the rate of change of delta (\(\Delta\)) with respect to changes in the underlying asset price (\(S\)). It represents the curvature or sensitivity of delta in response to movements in the underlying price. It provides insights into how rapidly an option's delta will change as the underlying price fluctuates.
\[
\Gamma = \frac{\partial^2 \text{Option Price}}{\partial S^2}
\]

For \textbf{knock-in} and \textbf{knock-out} barrier options, gamma exhibits unique behavior due to the presence of the barrier. The interaction between the price of the stock, the barrier level and the time to maturity influences how gamma behaves in different scenarios.

Figure~\ref{fig:gamma_behavior} illustrates the gamma behavior of an up-and-out put option as a function of the underlying stock price (\(S\)). 

\begin{figure}[h]
    \centering
    \includegraphics[width=.65\linewidth]{content/images/gamma-upout.png}
    \caption{Gamma behavior of an up-and-out Put Option vs. the Stock Price.}
    \label{fig:gamma_behavior}
\end{figure}

\section{Vega (\(\nu\))}

\begin{figure}[h]
    \centering
    \includegraphics[width=.65\linewidth]{content/images/vega_upout.png}
    \caption{Vega vs. Volatility of an Up-and-Out Put Option}
    \label{fig:vega_behavior}
\end{figure}

\textbf{Vega (\(\nu\))} measures the sensitivity of the option price to changes in volatility. For barrier options, the calculation of vega incorporates adjustments to account for the probability of breaching the barrier level under varying volatility conditions.
\[
\nu = \frac{\partial \text{Option Price}}{\partial \sigma}
\]

\begin{itemize}
    \item \textbf{Knock-In Options:} Increased volatility raises the likelihood of the underlying asset price crossing the barrier level, making the option more valuable. As a result, \(\nu\) (vega) will generally be positive.
    \item \textbf{Knock-Out Options:} Increased volatility can lead to a higher probability of the option being knocked out, reducing its value. Therefore, \(\nu\) will tend to be negative for these types of options.
\end{itemize}

The visualization shown in Figure~\ref{fig:vega_behavior} provides insights into how Vega behaves with changes in volatility across up-and-out options. The sensitivity patterns are influenced by the interplay between volatility and the probability of breaching the barrier level.


\section{4. Theta (\(\Theta\))}

Theta (\(\Theta\)) measures the rate of change of an option's price with the passage of time, assuming all other variables remain constant. It is a critical "Greek" that reflects time decay, which is the gradual erosion of the option's value as expiration approaches. In the case of barrier options, theta is influenced not only by the time remaining but also by the interaction with the barrier level and the path dependency of the option.

Barrier options are unique because their payoff depends on the underlying asset price crossing (or not crossing) a specified barrier level during the option's lifetime. As such, the sensitivity of theta changes depending on proximity to the barrier, time remaining until maturity, and whether the option is near the knock-in or knock-out condition.

\begin{figure}[h]
    \centering
    \includegraphics[width=.65\linewidth]{content/images/theta_upout.png}
    \caption{Theta vs. Time to Maturity of an Up-and-Out Put Option}
    \label{fig:theta_behavior}
\end{figure}

\begin{itemize}
    \item \textbf{Knock-In Options:} As time to maturity decreases, the value of knock-in options tends to decline, especially when the underlying asset price is far from the barrier. This is because the likelihood of triggering the option by crossing the barrier becomes lower as time runs out.
    
    \item \textbf{Knock-Out Options:} For knock-out options, time decay (\(\Theta\)) has a more pronounced effect, especially if the underlying asset price is close to the barrier. As time to maturity approaches, the risk of the underlying price reaching the barrier and knocking out the option increases, thereby accelerating time decay.
    
    \item \textbf{Proximity to the Barrier:} Theta becomes more sensitive when the underlying price is near the barrier level. This reflects the increased path dependency and likelihood of barrier activation.
\end{itemize}

The visualization in Figure~\ref{fig:theta_behavior} illustrates the relationship between time to maturity and theta for an up-and-out put option. This analysis highlights how theta's sensitivity varies depending on proximity to the barrier and the remaining time in the option's life. It is critical to consider these patterns when employing barrier options in trading strategies, especially as expiration approaches.


\section{Rho (\(\rho\))}

Rho for barrier options measures the sensitivity of the option price to changes in the risk-free interest rate (\(r\)). It represents how much the value of the option changes when the risk-free rate changes by 1\%. Rho is important in understanding the impact of monetary policy, such as changes in interest rates, on the value of barrier options.

\[
\rho = \frac{\partial \text{Option Price}}{\partial r}
\]

\begin{itemize}
    \item \textbf{Knock-In Options:} An increase in interest rates generally leads to a higher value for knock-in options. This is because higher rates make the option more attractive compared to other financial instruments by increasing the present value of the potential payoff.
    \item \textbf{Knock-Out Options:} The sensitivity of knock-out options to changes in interest rates is less pronounced compared to knock-in options, especially if the option is at a high risk of being knocked out before maturity.
\end{itemize}
\begin{figure}[h]
    \centering
    \includegraphics[width=.65\linewidth]{content/images/rho_upout.png}
    \caption{Rho vs. Interest Rate Changes for an Up-and-Out Put Option}
    \label{fig:rho_behavior}
\end{figure}

The visual behavior in Figure~\ref{fig:rho_behavior} depicts how the effect of interest rate changes impacts the value of an up-and-out barrier option under various interest rate conditions. When rates rise, knock-in options tend to show a positive relationship with Rho, reflecting the increased value from higher risk-free rates. Conversely, knock-out options tend to show muted sensitivity, as their value diminishes if the likelihood of being knocked out rises.
