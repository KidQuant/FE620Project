% !TEX root = ../thesis-example.tex
%
\chapter{Conclusion}

\section{Can Statistical Learning Help?}

Earlier, we wondered if methods that deliver the most accurate forecast at the individual stock level also provide the best-performing portfolio once we aggregate across stocks. As we can see from the previous section, the answer is not clear. Statistical methods that yield better predictions for individual stock returns, concerning $R^2_{oos}$, do not necessarily yield better portfolios when utilizing statistical financial metrics. One reason for this is the low signal-to-noise ratio in financial data.

An appropriate signal-to-noise ratio is achievable as alternatives to other features and data arise and machine learning techniques advance. At that time, the amount of data one needs to create accurate machine-learning models will decrease as the potential gain from these models will increase. However, until that time comes, low signal-to-noise ratios will remain an insurmountable challenge, which will make it empirically difficult to determine stock portfolios can be generated based on their statistical performance.