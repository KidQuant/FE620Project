% !TEX root = ../thesis-example.tex
%


\chapter{Introduction}
\everymath{\displaystyle}
\captionsetup[figure]{font=large}
Barrier options are path-dependent options with price barriers; their price depends on whether the underlying asset's price reaches a certain level during a specific period. Various types of barrier options regularly trade over-the-counter and have done so since 1967 \cite{JohnRubinstein1985}. These exotic options were developed to address the specific hedging concerns and market conditions that European and American options failed to accommodate. Barrier options are very popular for their risk-management solutions, as they allow investors and institutions to take various positions with very specific levels of protection.

As financial markets continued to evolve in the 1990s, barrier options became more standardized and accessible to the financial population. Derivative exchanges and financial institutions offered barrier options on various underlying assets, such as commodities, currencies, and interest rates. The 2008 global financial crisis sparked a renewed interest in derivative products for their risk-management capabilities, and barrier options remained one of the best products for their ability to tailor risk profiles to specific market conditions. In addition, computational tool advancements make pricing these options more manageable, thereby increasing the accessibility to market participants. This report will outline the many analytical and computational tools for pricing barrier options.